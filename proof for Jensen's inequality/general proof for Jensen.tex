\documentclass[10pt,a4paper]{article}

% typography %
\usepackage{amsmath, amsfonts, amssymb, amsthm}
\usepackage{mathtools} % e.g. mathclap
\usepackage{bbm}

% spacing %
\usepackage[pass]{geometry}
\usepackage{changepage} % e.g. adjustwidth 
\usepackage{setspace} % line spacing 

% urls %
\usepackage{hyperref}

% color %
\usepackage{xcolor} 
\hypersetup{
    colorlinks=false,
    linkcolor=black,   
    urlcolor=blue,
}


% tables & figures %
\usepackage{graphicx}
\usepackage{tikz, tikz-qtree}
\usepackage{slashbox} 
\usepackage{float} % positioning of tables and figures
\usepackage{caption} 
\captionsetup[table]{skip=2pt}

% misc %
\usepackage[makeroom]{cancel} % cancellation sign
\usepackage{authblk} % customise authors and affiliations

% codes %
\usepackage{listings}
\lstset{language=MATLAB,%
	basicstyle=\scriptsize\ttfamily,
	commentstyle=\ttfamily\color{gray},
	numbers=left,
	numberstyle=\ttfamily\color{gray}\footnotesize,
	stepnumber=1,
	numbersep=5pt,
	backgroundcolor=\color{white},    
    breaklines=true,%
    showstringspaces=false,%without this there will be a symbol in the places where there is a space
    frame = single,
    numbersep=5pt, % this defines how far the numbers are from the text
	caption={}
}

\lstset{
language=R,
basicstyle=\scriptsize\ttfamily,
commentstyle=\ttfamily\color{gray},
numbers=left,
numberstyle=\ttfamily\color{gray}\footnotesize,
stepnumber=1,
numbersep=5pt,
backgroundcolor=\color{white},
showspaces=false,
showstringspaces=false,
showtabs=false,
frame=single,
tabsize=2,
captionpos=b,
breaklines=true,
breakatwhitespace=false,
title=\lstname,
escapeinside={},
keywordstyle={},
morekeywords={},
caption={}
}

% code block %
\newenvironment{codeblock}{\ttfamily}{\par}
\def\code#1{\texttt{#1}} % inline %


% definitions %
\DeclareMathOperator*{\argmin}{argmin}
\DeclareMathOperator*{\argmax}{argmax}
\newcommand\norm[1]{\left\lVert#1\right\rVert}
\newcommand*\circled[1]{\tikz[baseline=(char.base)]{
            \node[shape=circle,draw,inner sep=2pt] (char) {#1};}}
\def\floor#1{\lfloor{#1}\rfloor}
\def\ceil#1{\lceil{#1}\rceil}
\def\liminfn#1{\liminf_{n \rightarrow #1}}
\def\limsupn#1{\limsup_{n \rightarrow #1}}
\def\limn#1{\lim_{n \rightarrow #1}}
\def\limN#1{\lim_{N \rightarrow #1}}

% definitions: theorems %
\theoremstyle{definition}
\newtheorem{theorem}{Theorem}
\newtheorem{corollary}{Corollary}[theorem]
\newtheorem{lemma}[theorem]{Lemma}\newtheorem*{remark}{Remark}\newtheorem*{notation}{Notation}
\newtheorem*{concept}{Concept}
\newtheorem*{outline}{Outline}
\newtheorem*{definition}{Definition}
\newtheorem{problem}{Problem}
\newtheorem{innercustomthm}{Theorem}
\newenvironment{customthm}[1]
  {\renewcommand\theinnercustomthm{#1}\innercustomthm}
  {\endinnercustomthm}


% definitions: probability %
\def\Xbd{\textbf{X}}
\def\Ybd{\textbf{Y}}
\def\P{\mathbbm{P}}
\def\E{\mathbbm{E}}
\def\Var{\text{Var}}
\def\eqd{\stackrel{d}{=}}

% definitions: theorems %
\def\bx{\boldsymbol{x}}
\def\by{\boldsymbol{y}}
\def\bz{\boldsymbol{z}}
\def\bf{\boldsymbol{f}}
\def\bA{\boldsymbol{A}}
\def\bAT{{\bA}^\top}
\def\bATA{{\bA}^\top \bA}
\def\bI{\boldsymbol{I}}
\def\bf{\boldsymbol{f}}

\def\hbx{\hat{\bx}}

\def\sig{\sigma}
\def\sigs{\sigma^2}
\def\hsig{\hat{\sigma}}
\def\tsig{\tilde{\sigma}}
\def\onesig{\frac{1}{\sigma}}
\def\onesigs{\frac{1}{\sigma^2}}
\def\onehsig{\frac{1}{\hsig}}
\def\onehsigs{\frac{1}{\hsig^2}}
\def\fsigs{f_{\sigma^2}}

\def\xkp{x_{k+1}}
\def\xk{x_k}
\def\xkm{x_{k-1}}
\def\xj{x_j}

\def\rmsekp{\textbf{RMSE}_{k+1}}
\def\rmsekm{\textbf{RMSE}_{k-1}}
\def\rmsek{\textbf{RMSE}_k}
\def\rmseo{\textbf{RMSE}_1}
\def\rmsef{\textbf{RMSE}_f}
\def\bP{\boldsymbol{P}}
\def\bQ{\boldsymbol{Q}}

\def\a{\alpha}
\def\l{\lambda}
\def\ve{\varepsilon}

\def\rf{\rho_f}
\def\bef{\boldsymbol{\varepsilon_f}}
\def\befhx{\bef(\hbx)}

\def\bxkp{\bx_{k+1}}
\def\bxk{\bx_k}
\def\bxkm{\bx_{k-1}}
\def\by{\boldsymbol{y}}
\def\bM{\boldsymbol{M}}

\def\bAT{\bA^\top}
\def\bATA{\bAT \bA}
\def\hby{\hat{\by}}

\def\bB{\boldsymbol{B}}

\def\sF{\mathcal{F}}
\def\sX{\mathcal{X}}
\def\sT{\mathcal{T}}
\def\sP{\mathcal{P}}
\def\sB{\mathcal{B}}
\def\sA{\mathcal{A}}

\def\sL{\mathcal{L}}

% defintions: special sets and functions %
\def\R{\mathbbm{R}}
\def\N{\mathbbm{N}}
\def\C{\mathbbm{C}}

\def\ind{\mathbbm{1}}

% special characters %
\def\os{ø}
 
% document-specific setting %

\setlength\parindent{0pt} % spacing

\graphicspath{{code/}} % image directory

\title{Optional notes on general proof for Jensen's inequality}
\author{Kevin H. Huang\footnote{PhD student, Gatsby Unit, UCL}}

\begin{document}

\newgeometry{left=1.0in, right=1.0in, top=1.2in, bottom=1.2in}

\onehalfspacing

\maketitle

\vspace{-20pt}

\textbf{Clarifications.} This is not the most general proof as we only consider real-valued random variable $X$ here. For the most general form with $X$ taking any values in a general topological space and with expectation allowed to be conditional expectations on any sub-sigma-algebra, one may refer to Wikipedia (\url{https://en.wikipedia.org/wiki/Jensen\%27s_inequality#General_inequality_in_a_probabilistic_setting}), which requires knowledge on topology and measure theory.

\ 

\noindent
\textbf{Credits:} This is adapted from James Norris's Year 1 Probability Notes for the Mathematics course at Cambridge (\url{http://www.statslab.cam.ac.uk/~james/Lectures/p.pdf}) by filling in some details for readers with a less mathematical background. We also extend the argument to $\R^d$ which is much cleaner to remember if one is familiar with convexity and subdifferentials. I am grateful to Yudong Chen for his corrections.

\section{1D case}

We start with some definitions and lemmas (which turn out to be alternative definitions) to set things up. We will see that Jensen's is straightforward once these are established.

\begin{definition}\textbf{(Integrability)}
A real-valued random variable $X$ is integrable if $\E(|X|) < \infty$.
\end{definition}

\begin{definition}\textbf{(Convexity in 1D)} A function $f: I \rightarrow \R$ defined on a convex set $I \subset \R$ is convex if, for any $x,y \in I$ and $t \in [0,1]$,
$$f\big(tx+(1-t)y\big) \leq tf(x) + (1-t)y.$$
\end{definition}

\begin{lemma} \label{alt_conv} \textbf{(Useful property (in fact alternative definition) of convexity)}
For a convex function $f: I \rightarrow \R$ defined on $I \subset \R$, given any $x, m, y \in I$ with $x < m < y$, we have
$$\frac{f(m)-f(x)}{m-x} \leq \frac{f(y)-f(m)}{y-m}.$$ 
\end{lemma}

\begin{remark}
\textit{Intuitively this says the ``local gradient" of the function is increasing or the ``local second derivative" is non-negative.}
\end{remark}

\begin{proof}
$x < m < y$ means there exists $t \in (0,1)$ such that $m=tx + (1-t)y$, so expressing $t$ in terms of $m$ and applying convexity,
\begin{align*}
f(m) &\leq \frac{y-m}{y-x} f(x) + \frac{m-x}{y-x} f(y) \\
\frac{y-m}{y-x} f(m) + \frac{m-x}{y-x} f(m) &\leq \frac{y-m}{y-x} f(x) + \frac{m-x}{y-x} f(y) \quad \text{ 
since }\frac{y-m}{y-x} + \frac{m-x}{y-x}  = 1 \\
\frac{y-m}{y-x} \big( f(m) - f(x) \big) & \leq \frac{m-x}{y-x} \big( f(y) - f(m) \big)
\end{align*}
Scaling both sides by $\frac{y-x}{(y-m)(m-x)}$ gives the result.
\end{proof}

\begin{lemma} \label{prop_conv}\textbf{(Another useful property (in fact alternative definition) of convexity)} For a convex function $f: I \rightarrow \R$ defined on \textbf{an open interval} $I \subset R$, given any $m \in I$, there exists $a, b \in \R$ such that
\begin{enumerate}
\item $am+b = f(m),$
\item $az+b \leq f(z)$ for any $z \in I.$
\end{enumerate}
\end{lemma}

\begin{remark}
\textit{Intuitively this says that a function has a ``supporting hyperplane" at any point $m \in I$. In fact in our extension to the n-dimensional case, and in the most general case on an abstract topology, this idea is characterised by replacing $a$ with any subgradient of $f$.}
\end{remark}

\begin{proof}
By Lemma \ref{alt_conv}, for any $x, y \in I$ with $x<m<y$ (which exist because $I$ is open), we have
$$\frac{f(m)-f(x)}{m-x} \leq \frac{f(y)-f(m)}{y-m},$$ 
therefore 
$$\sup_{x \in I, x \leq m}\frac{f(m)-f(x)}{m-x} \leq \inf_{y \in I, y \geq m}\frac{f(y)-f(m)}{y-m}.$$ 
So there exists $a \in \R$ such that
$$\sup_{x \in I, x \leq m}\frac{f(m)-f(x)}{m-x} \leq a \leq \inf_{y \in I, y \geq m}\frac{f(y)-f(m)}{y-m},$$ 
i.e. there exists $a \in \R$ independent of the choice of $x$ and $y$ such that
$$\frac{f(m)-f(x)}{m-x} \leq a \leq \frac{f(y)-f(m)}{y-m},$$ 
for all $x, y \in I$ with $x \leq m \leq y$. Rearranging we get
$a(z-m) + f(m) \leq f(z)$
for any $z \in I$ \textit{(for $z \leq m$ set $z=x$ above, for $z > m$ set $z=y$ above)}. Defining $b = f(m) - am$ gives us statement 1 and 2 in the lemma. 
\end{proof}

\begin{remark}
\textit{We have established that the convexity definition $\Rightarrow$ statement in Lemma \ref{alt_conv} $\Rightarrow$ statements in Lemma \ref{prop_conv}. The second statement in Lemma \ref{prop_conv} in fact imply the inequality in definition of convexity. One may see this by taking $m=tx+(1-t)y$ for $t \in (0,1)$, noting $b=f(m)-am$ to get $a(x-m) + f(m) \leq f(x)$ and $a(y-m) + f(m) \leq f(y)$, and finally taking a convex combination. For $t=0$ and $t=1$ it is trivial. Therefore, the statements in both lemmas are in fact alternative definitions of convexity. }
\end{remark}

\begin{theorem} (Jensen's inequality, 1D) For an integrable random variable $X$ taking values in an open interval $I \subset \R$, and a convex function $f: I \rightarrow \R$, we have
$$f(\E(X)) \leq \E(f(X)).$$
\end{theorem}
\begin{proof}
Take $m=\E(X)$ in Lemma \ref{prop_conv}, we have
$$f(\E(X)) \stackrel{\text{by 1 of Lemma }\ref{prop_conv} }{=} a\E(X)+b \stackrel{\text{by linearity of }\E}{=} \E(aX+b) \stackrel{\text{by 2 of Lemma }\ref{prop_conv}}{\leq} \E(f(X)).$$
\end{proof}

\begin{remark}
\textit{This is pretty trivial from Lemma \ref{prop_conv}, which we have seen is an alternative definition for convexity. In the case of strict convexity, one may show that statement 2 in Lemma \ref{prop_conv} becomes strict inequality for all $z \neq m$. The implication on Jensen's is that, for strictly convex function $f$, we have strict equality if and only if $X = \E(X)$ $\P$-almost surely, i.e. $X$ is constant with probability 1.}
\end{remark}

\section{$n$-dimensional case}



\begin{theorem} (Jensen's Inequality) For an integrable random variable $X$ taking values in an open interval $I \subset \R^d$, and a convex function $f: I \rightarrow \R$, we have
$$f(\E(X)) \leq \E(f(X)).$$
\end{theorem} 

\begin{proof}
$I$ is open so $\E(X)$ (in fact any point in $I$) is in the interior of $I$, which implies subdifferential of $f$ on that point is nonempty. Let $g$ be a subgradient of $f$ at $\E(X)$ then
\begin{align*}
f(\E(X)) &= g^{\top} \E(X) + f(\E(X)) -  g^{\top} \E(X) \\
&= \E\Big( g^{\top} X + f(\E(X)) -  g^{\top} \E(X) \Big) \\
&\leq \E\Big( g^{\top} X + f(X) -  g^{\top} X \Big) \\
& \qquad \text{ by definition of a subgradient at } \E(X) \text{ applied to the last two terms with respect to another point } X \\
&= \E(f(X))
\end{align*}
\end{proof}

\begin{remark} \textit{The similarity to the proof for 1D case is clear by matching $a$ with $g$ and $b$ with $f(\E(X)) -  g^{\top} \E(X)$.}
\end{remark}

\end{document} 


